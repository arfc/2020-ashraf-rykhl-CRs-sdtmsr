\section{Conclusion} \label{Conclusion}
In the current work, we introduced design of reactivity control system for the 
Single-fluid Double-zone Thorium-based Molten Salt Reactor (SD-TMSR). The 
design has been evaluated at the startup for the full-core geometry using 
SERPENT-2 Monte-Carlo code. We considered three various startup composition of 
the fuel salt: (1) $^{232}$Th as fertile and $^{233}$U as fissile material; 
(2) $^{232}$Th and plutonium, extracted from \gls{LWR} \gls{SNF}; (3) 
$^{232}$Th and transuranic vector, extracted from \gls{LWR} \gls{SNF}.

The excess reactivity $\rho_e$ is calculated at zero burnup when all CRs are 
fully withdrawn. The $\rho_e$ for $^{233}$U, reactor-grade Pu, and 
TRU are $1.65\pm0.04$ $\$$, $4.11\pm0.02$ $\$$, and $15.38\pm0.04$ $\$$, 
respectively.

Six different absorbing materials are considered in this work:
natural B$_4$C, B$_4$C-90 (boron is enriched to 90\% $^{10}$B), HfB$_2$, 
HfH$_{1.62}$, Eu$_2$O$_3$, and Gd$_2$O$_3$. Enriched B$_4$C-90 has the largest 
absorption ability, while Gd$_2$O$_3$ has the lowest absorption compared with 
the other absorbing materials in this study. Both CSD and SSD clusters are 
separately able to shut down the reactor initially loaded with $^{233}$U and 
reactor-grade Pu regardless of the absorbing material type. However, only SSD 
clusters made of B$_4$C-90 are able to shut down the SD-TMSR initially loaded 
with transuranic vector from \gls{LWR}. The reason for this is much larger 
absorption cross section of $^{10}$B in the relatively soft neutron energy 
spectrum of the SD-TMSR core started with TRU.

In case of malfunction of the other CR clusters (e.g., stuck in the upper 
position), the outer ring of the CSD failed to counteract the excess 
reactivity of the core initially loaded with reactor-grade Pu and transuranic 
(TRU) elements. However, the worth of the outer ring of the CSD is sufficient 
to compensate the excess reactivity for the core refueled by $^{233}$U.

All absorbing materials provide an adequate shutdown margin for the SD-TMSR 
core that initially loaded with $^{233}$U and reactor-grade Pu. However, the 
shutdown margins for TRU case are negative or slightly positive (in B$_4$C-90 
case), this makes the SSD clusters unreliable to shut down the reactor loaded 
with TRU because the SSD clusters should provide a sufficient positive shutdown margin.

The amplification factor (A$_{CRi}$) results show that the CSD, SSD, CSD inner ring, and SSD1 are slightly amplified due to the anti-shadowing effects. The strongest anti-shadowing effect has observed in SSD4 and CSD9 clusters that are located at the boundary between the core zones with different moderator-to-fuel ratio.

The insertion of CRs deforms the radial flux shape in certain positions, i.e., around CRs positions. This shifts the neutron flux from the core center towards the periphery.

The integral and differential control rod worth are calculated for three 
different systems: all control rods, CSD, and SSD systems. The results show 
that all absorbing materials have almost the same integral rod worth in the 
upper half of the core. Further insertion of the control rods shows the unique 
absorption characteristics of each material.

Finally, the proposed design of CRs successfully control the excess reactivity and enhance the safety aspects of the SD-TMSR.

\section{Future work}
In Part II of this research effort, we will evaluate effect of fuel salt 
burnup on the CRW. The depleted fuel composition will be obtained using 
SERPENT-2 online reprocessing subroutine \cite{aufiero2013extended} and 
batch-wise tool SaltProc \cite{rykhlevskii_arfc/saltproc_2018, 
rykhlevskii_milestone_2019}. Additionally, the authors intend to investigate 
kinetic parameters (effective delayed neutron fraction $\beta_{eff}$ and 
effective delayed neutron precursor decay constant $\lambda_{eff}$) evolution 
during the SD-TMSR operation. These parameters are crucial for accident 
transient analysis, which will be performed using Moltres 
\cite{lindsay_introduction_2018}, multi-physics code for liquid-fueled MSR 
simulation with taking into account the neutron precursors drift.

\section{Declaration of Competing Interest}

The authors declare that they have no known competing financial interests or personal relationships that could have appeared to influence the work reported in this paper.
