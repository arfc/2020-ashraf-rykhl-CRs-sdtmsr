\section{Conclusion} \label{Conclusion}
The full-core of the Single-fluid Double-zone Thorium-based Molten Salt Reactor (SD-TMSR) is loaded by three different types of initial fissile materials: $^{233}$U, reactor-grade Pu, and transuranic (TRU) elements from \gls{LWR} \gls{SNF}. The excess reactivity $\rho$$_e$ is calculated at zero burnup when all CRs are fully withdrawn. The excess reactivities for $^{233}$U, reactor-grade Pu, and TRU are $1.65\pm0.04$ $\$$, $4.11\pm0.02$ $\$$, and $15.38\pm0.04$ $\$$, respectively.

Five different absorber materials are considered based on their neutronics and safety performance: B$_4$C-90, HfB$_2$, HfH$_{1.62}$, Eu$_2$O$_3$, and Gd$_2$O$_3$. Enriched B$_4$C-90 has the largest absorption ability, while Gd$_2$O$_3$ has the lowest absorption compared with the other absorber materials in this study. Both CSD and SSD clusters are separately able to shut down the reactor initially loaded by $^{233}$U and reactor-grade Pu regardless of the absorber material type. However, only SSD clusters made of B$_4$C-90 is able to shut down the SD-TMSR initially loaded by transuranic (TRU) elements.

In case of malfunction of the other CR clusters (e.g. stuck in the upper position), the outer ring of the CSD failures to counteract the excess reactivity of the core initially loaded by reactor-grade Pu and transuranic (TRU) elements. However, the worth of the outer ring of the CSD is sufficient to compensate the excess reactivity for the core refueled by $^{233}$U.

All absorber materials provide an adequate shutdown margin for the SD-TMSR core that initially loaded by $^{233}$U and reactor-grade Pu. However, the shutdown margins for TRU case are negative or slightly positive (in B$_4$C-90 case), this makes the SSD clusters ineffective to shut down the reactor loaded by TRU.

The amplification factor (A$_{CRi}$) results show that the CSD, SSD, CSD inner ring, and SSD1 are slightly amplified due to the anti-shadowing effects. The strongest anti-shadowing effect has observed in SSD4 and CSD9 clusters that are located at the boundary between the core zones with different moderator-to-fuel ratio.

The insertion of CRs deforms the radial flux shape in certain positions, i.e., around CRs positions. This shifts the neutron flux from the core center towards the periphery.

The integral and differential control rod worth are calculated for three different systems: all control rods, CSD, and SSD systems. The results show that all absorber materials have almost the same integral rod worth in the upper quarter of the core ($x$$<$$130$). Further insertion of the control rods shows the unique absorption characteristics of each material.

\section{Future work}
The effect of burnup on the absorption ability of control rods will be investigated specifically in the second part of this paper. Additionally, .......

\section{Declaration of Competing Interest}

The authors declare that they have no known competing financial interests or personal relationships that could have appeared to influence the work reported in this paper.
