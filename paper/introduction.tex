\section{Introduction} \label{Introduction}

The \gls{GIF} \cite{doe2002technology} determined six innovative reactor 
systems: the Very High-Temperature Reactor (VHTR), the \gls{MSR}, the 
Supercritical Water-Cooled Reactor (SWCR), the Gas-cooled Fast Reactor (GFR), 
the Sodium-cooled Fast Reactor (SFR), and the Lead-cooled Fast Reactor (LFR). 
The MSR is the only liquid-fueled reactor among these reactors. Major nuclear 
centers pursue MSRs with renewed interest \cite{betzler_impacts_2019, 
ashraf2020whole,betzler2016modeling,mohsin2019safety,zhang2020radiotoxicity}.
The MSR is a promising technology because of its dynamics, especially for 
thorium fuel utilization. However, this technology has difficulties in safety, 
online reprocessing, and fuel handling. Safety is one of the primary goals of 
the MSR systems, which was understudied and require further investigation. The 
unique characteristics of the MSRs (liquid fuel, flux level, neutron economy, 
etc.) strongly affect its control system design. Graphite-moderated MSRs have 
a negative total temperature coefficient of reactivity 
\cite{ashraf2020whole,robertson_conceptual_1971,nuttin2005potential,rykhlevskii2019modeling,li_optimization_2018}.
Doppler effect and thermal expansion of the fuel reduce core reactivity when 
the core heats up. MSR designs have a drain tanks to 
hold and cool down the liquid fuel in an emergency case. A freeze plug, which 
located under the core, passively melts when the fuel temperature reaches a 
critical point, and drains the fuel salt from the reactor vessel to the drain 
tanks. The drain tanks have a subcritical
configuration with a large free 
surface area to readily dissipate heat by passive cooling 
\cite{elsheikh2013safety}.

At the Beginning of Life (BOL), the MSR is commonly load with larger amount of 
fuel than that required to achieve criticality (necessary for long term core 
operation). This leads to excess reactivity at the BOL. Recent studies on MSRs showed that this excess reactivity 
at the BOL is quite large for many refueling scenarios 
\cite{ashraf2020Strategies,ashraf2020whole,rykhlevskii2019modeling,betzler2016modeling,ashraf2018nuclear,ashraf2019modeling}.
Additionally, during burnup, the online 
reprocessing and refueling leads to slow reactivity change, which 
should be compensated \cite{ashraf2020Strategies,ashraf2020whole}. 
The online reprocessing and refueling system helps to control the reactivity of 
the MSRs \cite{ashraf2020whole}. However, the most common and reliable procedure 
for reactor control is to insert or withdraw control rods made of material 
with large neutron absorption cross section (e.g., boron, cadmium) 
\cite{duderstadt650nuclear}. 
Insertion of control rods introduces an amount of negative reactivity into the 
core. This negative reactivity helps to compensate the excess reactivity and 
adjust the power level of the core or shut down the reactor in case of 
emergency \cite{glasstone1967nuclear}. Therefore, we should estimate 
the reactivity worth of the control rods 
\cite{varvayanni2009estimation,fadaei2009control,aoyama2007core,bretscher1997computing}.
 The reactivity worth of control rods correlates with the interference 
(shadowing effects) between control rod clusters 
\cite{girardin2008development,vcerba2017optimization}. 
The control rod worth and its efficiency to absorb excess positive reactivity is a subject of major interest since it directly affects reactor safety \cite{liu2018criticality,atkinson2019small,vcerba2017optimization,do2019criticality,guo2019advanced,varvayanni2009estimation}.

Boron carbide (B$_4$C) is commonly used material for the control rods 
\cite{zhong2019preliminary,steinbruck2010degradation,dunner1984absorber}. 
However, we may need to enrich the boron isotope ($^{10}$B) to reach 
the necessary absorption cross section. Additionally, issues related to helium gas release through (n,$\alpha$) reactions of $^{10}$B, swelling,
and high loss of the reactivity worth limit the B$_4$C lifespan 
\cite{guo2019optimized}. Guo (2019), Gosset (2017), and Rudy (2011) summarized the properties of the 
potential alternative absorbers for Generation-IV reactors such as 
hafnium-based materials and rare earth element oxides. These absorbers have 
high thermal conductivity, good resistance to neutron irradiation, and release 
almost no gas (absorbing neutrons mainly through (n,$\gamma$) reactions)
\cite{guo2019optimized,gosset2017absorber,konings2011comprehensive}.

The \gls{SD-TMSR} with a thermal power of 2,250 MW$_{th}$ 
\cite{ashraf2020whole,li_optimization_2018} is a graphite-moderated 
molten salt reactor. Adjusting fertile and fissile feed rates helps to control the reactivity of 
the SD-TMSR \cite{ashraf2020whole,ashraf2020Strategies,li_optimization_2018}. However, reactivity control system for maneuvering and emergency shutdown was not introduced in the literature \cite{li_optimization_2018,zou2018transition,zhang2020radiotoxicity,jiang2012advanced,zou2018preliminary,ZOU2015114}. Therefore, the main objective of our study is to introduce a new 
fast reactivity control system based on control rods in the \gls{SD-TMSR} to control 
the reactivity during normal operation and shut down the reactor in case of 
emergency. Five different absorbing materials are 
applied to study the main neutronics and safety parameters in the SD-TMSR. We 
focus on the control rod design, absorption ability, integral and differential 
control rod worths, shutdown margin of absorbing materials, and shadowing 
effects at steady state calculation.

\v{C}erba \emph{et al.} (2017) designed a reactivity control system for the 
GFR. They utilized the MCNP5 \cite{briesmeister2000mcnptm} and KENO6 codes 
\cite{petrie1984keno} to calculate the reactivity worth and analyze the 
performance of this control system \cite{vcerba2017optimization}. Since no 
final geometry design of the control rods system is available for the SD-TMSR, 
\v{C}erba's methodology \cite{vcerba2017optimization} helped us as a starting 
point of this analysis.

All calculations presented in this work are performed using Monte-Carlo code SERPENT-2 version 2.1.31 \cite{leppanen2014serpent}.

This paper is organized as follows: section \ref{Model-description} discusses the reactor and control rod design, section \ref{Methodology-and-tools} describes methodology and tools, section \ref{Results-and-discussion} focuses on the results and discussion, and section \ref{Conclusion} highlights the conclusions.