\section{Introduction} \label{Introduction}

The \gls{GIF} \cite{doe2002technology} determined six innovative reactor 
systems for further research and commercialization: the Very High-Temperature Reactor (VHTR), the \gls{MSR}, the 
Supercritical Water-Cooled Reactor (SWCR), the Gas-cooled Fast Reactor (GFR), 
the Sodium-cooled Fast Reactor (SFR), and the Lead-cooled Fast Reactor (LFR). 
The MSR is the only liquid-fueled reactor among these reactors. Major nuclear 
centers are pursuing MSRs with renewed interest \cite{betzler_impacts_2019, 
ashraf2020whole}
due to its potential advantages, for example, thorium fuel utilization, small waste production, and ability to use spent fuel.
However, this technology has difficulties including safety, 
online reprocessing, and fuel handling. Safety is one of the primary goals of
the MSR system, and the SD-TMSR, in particular, requires further investigation to support a licensing case.
The unique characteristics of the MSR (liquid fuel, flux level, neutron economy, 
etc.) strongly affect its control system design. Doppler effect and thermal expansion of the fuel reduce
core reactivity when the core heats up, thus MSRs have negative total temperature coefficient of reactivity 
\cite{nuttin2005potential}.
MSR designs have drain tanks to 
hold and cool down the liquid fuel in an emergency. In some MSR design concepts, a freeze plug, which 
is located under the core, passively melts when the fuel temperature reaches a 
critical point, and drains the fuel salt from the reactor vessel to the drain 
tanks. The drain tanks have a subcritical
configuration with a large free 
surface area to readily dissipate heat by passive cooling 
\cite{elsheikh2013safety}.

At the Beginning of Life (BOL), MSRs are commonly loaded with more
fuel than that required to achieve criticality (necessary for long-term core 
operation); this leads to excess reactivity at the BOL. Robertson (1971) assumed small excess reactivity for \gls{MSBR} loaded by $^{233}$U/$^{232}$Th fuel, thus only 2 control (graphite) rods for adjusting reactivity and 2 safety (B$_4$C) rods for emergency shutdown were suggested \cite{robertson_conceptual_1971}. However, scientists are concerned about $^{233}$U supply, since it does not exist anywhere. Our recently published paper \cite{ashraf2020Strategies} concluded that for realistic fueling scenarios (e.g., TRU/$^{232}$Th and Pu/$^{232}$Th), large excess reactivity is required. Betzler \emph{et al.} (2017) also reported large excess reactivity to operate Transatomic Power Molten Salt Reactor (TAP MSR) for a long time \cite{betzler2017assessment}. Additionally, Rykhlevskii \emph{et al.} (2019) studied the fuel cycle performance for various designs of the \gls{MSFR}. For MSFRs, a considerable excess reactivity is also required for long-term core operation \cite{rykhlevskii_fuel_2019}. Recent studies on MSRs showed that the excess reactivity 
at the BOL may vary from 1.2 \% for $^{233}$U-loaded core \cite{rykhlevskii2019modeling,betzler2016modeling} to 8.1 \% for Pu/TRU-loaded core \cite{ashraf2020Strategies}.

The online reprocessing and refueling system is designed to operate as long-term reactivity control in the MSRs \cite{ashraf2019modeling,ashraf2019Preliminary}. However, we cannot rely on online reprocessing to adjust reactivity quickly, because loop time (time needed for one particle to make one full circulation through the primary loop) is approximately 20 seconds for MSBR and its clones. Consequently, we need a prompt reactivity control system for emergencies. The most common and reliable procedure to control the reactor is to insert or withdraw control rods made of material with a large neutron absorption cross section (e.g., boron).
Insertion of control rods introduces negative reactivity into the 
core which helps to compensate the excess reactivity and 
adjust the power level of the core or shut down the reactor in case of 
emergency. Therefore, we should estimate 
the reactivity worth of the control rods.
The reactivity worth of control rods correlates with the interference 
(shadowing effects) between control rod clusters 
\cite{vcerba2017optimization}. 
The control rod worth and its efficiency to absorb excess positive reactivity is a subject of major interest since it directly affects reactor safety \cite{atkinson2019small}. Additionally, Nuclear Regulatory Commission (NRC) regulation requires that one of the reactivity control systems use control rods \cite{nuclear1987standard}.

Boron carbide (B$_4$C) is a commonly used material for the control rods; 
however, we may need to enrich the boron isotope ($^{10}$B) to reach 
the necessary absorption cross section. Additionally, issues related to helium gas release through (n,$\alpha$) reactions of $^{10}$B, high loss of the absorption ability under irradiation, and swelling limit the B$_4$C lifespan 
\cite{guo2019optimized}. Guo (2019), Gosset (2017), and Rudy (2011) summarized the properties of the 
potential alternative absorbers for Generation-IV reactors such as 
hafnium-based materials and rare earth oxides. These absorbers have 
high thermal conductivity, good resistance to neutron irradiation, and
absorb neutrons mainly through (n,$\gamma$) reactions
\cite{guo2019optimized}.

The SD-TMSR with a thermal power of 2,250 MW$_{th}$ 
\cite{li_optimization_2018} is a graphite-moderated 
molten salt reactor. Adjusting fertile and fissile feed rates helps to control the reactivity of 
the SD-TMSR \cite{ashraf2020Strategies,li_optimization_2018}. However, a reactivity control system for maneuvering and emergency shutdown in the SD-TMSR has not been introduced in the literature \cite{li_optimization_2018,zou2018transition,zhang2020radiotoxicity}. Therefore, the main objective of our study is to introduce a new, 
rapid reactivity control system based on control rods in the SD-TMSR to control 
the reactivity during normal operation and shut down the reactor in case of 
emergency. Six absorbing material options are considered in the context of the 
neutronics and safety parameters in the SD-TMSR. We 
focus on control rod design, absorption ability, integral and differential 
control rod worths, shutdown margin, and shadowing 
effects at steady state calculation.

Xuemei \emph{et al.} (2013) calculated the control worth in the thorium molten salt
reactor by using MCNP \cite{briesmeister2000mcnptm}. We adopted the SD-TMSR
core geometry optimized by Li \emph{et al.} \cite{li_optimization_2018} which
is different from the Molten Salt Breeder Reactor-like geometry by Mathieu \emph{et al.}
\cite{mathieu2006thorium} adopted by Xuemei \emph{et al.} \cite{xuemei2013study}.
Additionally, Xuemei \emph{et al.} studied the natural B$_4$C as absorbing material
while we considered six different absorbing materials.

\v{C}erba \emph{et al.} (2017) designed a reactivity control system for the 
GFR. They utilized the MCNP \cite{briesmeister2000mcnptm} and KENO6 codes 
\cite{petrie1984keno} to calculate the reactivity worth and analyze the 
performance of this control system \cite{vcerba2017optimization}. Since no
geometry design of the control rods system is available for the SD-TMSR, 
\v{C}erba's methodology \cite{vcerba2017optimization} helped us as a starting 
point of this analysis.

The online reprocessing and refueling is a distinguishing feature for MSRs comparing with traditional solid fuel reactors. Its neutron characters including $k_{eff}$, excess reactivity, Breeding Ratio (BR), and $^{233}$U production are totally different when operating from those of startup. The $k_{eff}$, excess reactivity, and neutron flux are all related to the scheme of reprocessing. Therefore, the worth of the control rods should be corresponding to the reprocessing scheme; that is, the rate of extraction and addition of elements. However, the present work focuses on the steady state calculations (i.e. without burnup).

All calculations presented in this work are performed using Monte-Carlo code SERPENT-2 version 2.1.31 \cite{leppanen2014serpent}.

This paper is organized as follows: section \ref{Model-description} discusses the reactor and control rod design, section \ref{Methodology-and-tools} describes methodology and tools adopted to evaluate the control rod design, section \ref{Results-and-discussion} focuses on calculated control rod parameters such as integral and differential worth, shutdown margin (SDM), and amplification factors, and section \ref{Conclusion} highlights the conclusions.