\begin{abstract}
\gls{MSR} reactivity control system design has historically been conceptual 
rather than concrete. Recent studies on \glspl{MSR} showed that the excess reactivity 
at the beginning of the operation is large for many fueling strategies 
and must be compensated by a reactivity control system. The current work 
introduces a reliable safety system based on control rods in addition to the 
online feed system reactivity control in the 
\gls{SD-TMSR}. Three different initial fissile loadings are considered: $^{233}$U, reactor-grade Pu, and transuranic (TRU) elements as a startup fuel. We applied six different absorbing 
materials to investigate the main operational and safety parameters in the 
\gls{SD-TMSR}: natural B$_4$C, enriched B$_4$C with 90\% $^{10}$B, HfB$_2$, HfH$_{1.62}$,
Eu$_2$O$_3$, and Gd$_2$O$_3$.
The present work focuses on control rod design, integral and differential 
control rod worth, shutdown margin, and shadowing effects at steady state. 
We employed the SERPENT-2 Monte-Carlo code to calculate the reactivity worth and 
analyze the performance of the reactivity control system. We showed that $^{233}$U 
and reactor-grade Pu startup cores maintain adequate shutdown margin with all considered absorbers. Finally, this paper proposes a design of control rod clusters that compensate the excess reactivity of the SD-TMSR loaded with different initial fissile material.
\end{abstract}