\begin{abstract}
\glspl{MSR} reactivity control system design has historically been conceptual 
rather than concrete. The most common way to control reactivity is to move 
control rods. Recent studies on \glspl{MSR} showed that the excess reactivity 
at the beginning of the operation is large for many fueling strategies 
and must be compensated by reactivity control system. The current work 
introduced a reliable safety system based on control rods in addition to the 
online feed system reactivity control in the 
\gls{SD-TMSR}. The full-core of the \gls{SD-TMSR} model contained fuel salt 
with three different initial fissile materials: $^{233}$U, reactor-grade Pu, 
and transuranic elements (TRU) as a startup fuel. We applied five different absorbing 
materials to investigate the main operational and safety parameters in the 
\gls{SD-TMSR}: B$_4$C, HfB$_2$, HfH$_{1.62}$, Eu$_2$O$_3$, and Gd$_2$O$_3$. 
Present work focused on the control rod design, integral and differential 
control rod worth, shutdown margin, and shadowing effects at the steady state. 
We employed SERPENT-2 Monte-Carlo code to calculate the reactivity worth and 
analyze the performance of the reactivity control system. For the $^{233}$U 
and reactor-grade Pu startup cases all considered absorbers demonstrated an 
adequate shutdown margin. However, for the TRU case, the shutdown margin is 
negative or slightly positive, which makes the Shutdown Safety Devices (SSD)
unreliable for shutting down the reactor. Finally, the proposed design of the 
control rod clusters showed the possibility to compensate the excess 
reactivity of the SD-TMSR core loaded with different initial fissile material.
\end{abstract}