\section{Results and discussion} \label{Results-and-discussion}

\subsection{The excess reactivity $\rho$$_e$}

The excess reactivity $\rho$$_e$ is calculated at zero burnup (steady state 
calculation), when all CRs are fully withdrawn. $\rho_e$ for $^{233}$U, 
reactor-grade Pu, and TRU used as initial fissile material are $1.65\pm0.04$ 
$\$$, $4.11\pm0.02$ $\$$, and $15.38\pm0.04$ $\$$, respectively. For $^{233}$U 
case, the maximum excess reactivity $\rho$$_e$ is about $4.27\pm0.01$ $\$$ 
during 60 effective full-power years (EFPY) of reactor operation (see 
Figure~\ref{fig:keff_25}). 
The SD-TMSR is able to control the reactivity by adjusting the online 
refueling and reprocessing rates \cite{ashraf2019whole_core}. An effective and 
reliable reactivity control system must be designed beside the online feed 
system to operate with such reactivity changes during burnup.

\begin{figure}
	\centering
	\includegraphics[width=\textwidth]{keff_25.png}
	\vspace{-0.5in}
	\caption{The change of the effective multiplication factor during 60 EFPY of reactor operation including periodic fissile material insertion (confidence interval $\pm\sigma$ is shaded) \cite{ashraf2019whole_core}.} 
	\label{fig:keff_25}
\end{figure}

\subsection{Control rod parameters}

The control rod parameters including control rod worth (CRW), interference 
between CR clusters, and integral and differential control rod worths are 
described in this part. Five different absorbing materials are considered based 
on their neutronics and safety performance (see part \ref{CRD}). 
Table~\ref{tab:worth} shows calculated control rod worth, the amplification 
factor (A$_{CRi}$), and the type of interference for five different 
absorbers.

\begin{sidewaystable}
	\fontsize{6}{9}\selectfont
	\centering
	\caption{The control rod worth and shadowing effect for different CR 
	materials.}
	\vspace{1ex}
	\begin{tabularx}{\textwidth}{|X|p{0.65cm}|p{0.65cm}|p{0.65cm}| 
	p{0.65cm}|p{0.65cm}|p{0.65cm}| p{0.65cm}|p{0.65cm}|p{0.65cm}| 
	p{0.65cm}|p{0.65cm}|p{0.65cm}| p{0.65cm}|p{0.65cm}|p{0.65cm}|}
		\hline
		\multirow{2}{*}{Control Rod group}		& 
		\multicolumn{3}{c|}{B$_4$C-90}   	&\multicolumn{3}{c|}{HfB$_2$}	
		&\multicolumn{3}{c|}{HfH$_{1.62}$} 
		&\multicolumn{3}{c|}{Gd$_2$O$_3$}	& 	
		\multicolumn{3}{c|}{Eu$_2$O$_3$} \\
		\cline{2-16}
		& $\Delta\rho$$_{CRi}$  [\$]  &A$_{CRi}$	& Interfe\-rence	
		&$\Delta\rho$$_{CRi}$ [\$]&A$_{CRi}$	& Interfe\-rence	
		&$\Delta\rho$$_{CRi}$ [\$]	&A$_{CRi}$	& Interfe\-rence	
		&$\Delta\rho$$_{CRi}$ [\$]	&A$_{CRi}$	& Interfe\-rence	
		&$\Delta\rho$$_{CRi}$ [\$]	&A$_{CRi}$  &	Interfe\-rence \\
		\hline                   
		All control rods        				& $48.19\pm0.68$   &	&	
		&$40.4\pm0.41$	&	&	&$37.96\pm0.36$	&	&	&$33.7\pm0.4$	&	
		&	&$42.39\pm0.48$	& &	 \\
		\hline 
		CSD 							& $25.5\pm0.23$   &$1.24\pm0.01$	&$\star$	&$21.9\pm0.17$	&$1.1\pm0.01$	&$\star$	&$20.62\pm0.34$	&$1.1\pm0.01$	&$\star$	&$18.48\pm0.25$	&$1.17\pm0.01$	&$\star$	&$22.96\pm0.18$	&$1.2\pm0.01$ &$\star$ \\
		\hline 
		SSD		  		&$16.39\pm0.11$ &$1.38\pm0.02$	&$\star$	&$14.27\pm0.21$	&$1.29\pm0.01$	&$\star$	&$13.23\pm0.18$	&$1.3\pm0.06$	&$\star$	&$12\pm0.2$	&$1.26\pm0.01$	&$\star$	&$14.58\pm0.11$	&$1.3\pm0.02$	&$\star$ \\
		\hline 
		CSD inner ring  &  $18.42\pm0.05$  &$1.6\pm0.04$	&$\star$		&$16.29\pm0.29$	&$1.48\pm0.02$	&$\star$	&$15.5\pm0.17$	& $1.46\pm0.01$ &$\star$	&$14.12\pm0.07$	&$1.39\pm0.01$	&$\star$	&$16.92\pm0.13$	&$1.5\pm0.01$	 &$\star$\\
		\hline 
		CSD outer ring            	    &$2.26\pm0.02$ &$4.21\pm0.07$	&$\star$	&$2.1\pm0.06$	&$3.11\pm0.1$&$\star$	&$1.85\pm0.07$	&$3.24\pm0.06$	&$\star$	&$1.8\pm0.1$	&$2.54\pm0.14$	&$\star$	&$2.14\pm0.05$	&$3.2\pm0.05$	 &$\star$	\\  
		\hline 
		CSD2								& $2.25\pm0.04$   &$3.69\pm0.16$	&$\star$	&$2.24\pm0.05$	&$2.42\pm0.1$	&$\star$	&$2\pm0.1$	&$2.68\pm0.03$	&$\star$	&$1.8\pm0.1$	&$2.55\pm0.2$	&$\star$	&$2.19\pm0.12$	&$2.9\pm0.08$	&$\star$ \\
		\hline 
		CSD9								& $0.15\pm0.04$   & $10.53\pm0.01$	&$\star$$\star$	&$0.1\pm0.05$	&$5.1\pm0.05$	&$\star$	&$0.05\pm0.01$	&$16.4\pm0.1$	&$\star$$\star$	&$0.09\pm0.07$	&$5.55\pm0.2$	&$\star$$\star$	&$0.07\pm0.05$	&$14\pm0.1$	&$\star$$\star$ \\ 
		\hline
		SSD1					& $4.18\pm0.07$   &$1.59\pm0.08$	&$\star$	&$3.95\pm0.13$	&$1.3\pm0.07$	&$\star$	&$3.59\pm0.03$	&$1.49\pm0.01$	&$\star$	&$3.45\pm0.12$	&$1.4\pm0.08$	&$\star$	&$3.91\pm0.06$	&$1.4\pm0.01$&$\star$ \\
		\hline 
		SSD4					&  $0.57\pm0.09$  &$7.84\pm0.15$	&$\star$$\star$	&$0.63\pm0.06$	&$4.4\pm0.1$	&$\star$	&$1\pm0.05$	&$2.51\pm0.16$	&$\star$	&$0.6\pm0.8$	&$3.31\pm0.27$	&$\star$	&$0.6\pm0.08$	&$5.1\pm0.28$	&$\star$$\star$ \\
		\hline
	\end{tabularx}
	\begin{tablenotes}
		\tiny
		\item  $\star$  anti-shadowing effects observed
		\item  $\star$$\star$ strong anti-shadowing effects observed
	\end{tablenotes}
	\label{tab:worth}
\end{sidewaystable}

\subsubsection{CRW} \label{CR_worth}

The total worth of all control rods ranges from $33.7\pm0.4$ to $48.19\pm0.68$ 
\$ (Table~\ref{tab:worth}). B$_4$C-90 (boron is enriched to 90\% $^{10}$B) 
has the largest absorption ability, while Gd$_2$O$_3$ has the lowest 
absorption compared with the other absorbing materials in this study. This 
result agrees with macroscopic absorption cross sections data 
\cite{guo2019optimized}. B$_4$C-90 has the highest macroscopic 
absorption cross sections followed by Eu$_2$O$_3$, HfB$_2$, HfH$_{1.62}$, and 
finally Gd$_2$O$_3$. The control rod material are being transmuted during 
operation. The effect of the fuel salt burnup on the control rod worth is not 
considered herein, and will be investigated and presented in Part II of this 
study.

Additionally, the worth of the CSD clusters is by factor $1.56$ greater than 
the worth of the SSD system for all absorbing materials. Either CSD or SSD 
clusters are able to shut down the reactor initially loaded by 
$^{233}$U and reactor-grade Pu regardless of the absorbing material type. 
However, only SSD clusters made of B$_4$C-90 are able to shut down the SD-TMSR 
initially loaded with transuranic (TRU) elements. Increasing the number of SSD 
cluster or changing their location would increase the worth.

The inner ring of the CSD is located in the central zone of the SD-TMSR core 
(Figure~\ref{fig:core_25}), where the volume ratio between molten salt and 
graphite is $0.357$. Results show that the inner ring of the CSD has 
the worth almost equal to the worth of all other CRs together regardless of 
the absorbing material type (Table~\ref{tab:worth}). This may be attributed to 
the fact that the absorption cross section decreases with the energy of 
incident neutron, for example, boron absorbs neutrons in thermal spectrum much 
greater than in fast spectrum.

In case of malfunction of other CR clusters (e.g., stuck in the upper 
position), the outer ring of the CSD cannot compensate the excess 
reactivity of the core initially loaded by reactor-grade Pu and transuranic 
(TRU) elements. However, the outer ring of the CSD worth is sufficient 
to compensate the excess reactivity for the core refueled by $^{233}$U . 

We separately calculated the worth of CSD2, CSD9, SSD1, and SSD4 clusters to 
investigate the variation of CRW with the position in the active core.
The CRW decreases in the direction of the outer core zone. The outer core zone 
has smaller moderator-to-fuel ratio ($0.86$) compared with the central zone 
($2.8$), consequently, the neutron energy spectrum is faster in the peripheral 
zone than in the center of the core. As mentioned previously, the CRs 
absorption ability degrades in fast spectrum.

\subsubsection{Shutdown Margin (SDM)}

The SSD clusters are designed mainly for an emergency shutdown, thus it should 
provide the reactor core with sufficient and adequate negative reactivity. The 
shutdown margin (SDM) is calculated by equation~\ref{Equ:6}.  
Table~\ref{tab:table2} summarizes the shutdown margins for the SD-TMSR core 
initially loaded with $^{233}$U,  reactor-grade Pu, and transuranic (TRU) 
elements for different absorbing materials. All absorbing materials provide an 
adequate shutdown margin for the SD-TMSR core that initially loaded with 
$^{233}$U and reactor-grade Pu. However, the shutdown margins for TRU case are 
negative or slightly positive (in B$_4$C-90 case), this makes the SSD clusters 
ineffective to shut down the reactor loaded with TRU.
\begin{table}  [!hb]
	\caption{The shutdown margins for the SD-TMSR core for different absorbing materials.}
	\vspace{0.1in}
	\begin{tabularx}{\textwidth}{p{3cm} s s s}
		\hline
		Absorbing materials        				&  $^{233}$U & reactor-grade Pu&  TRU \\
		\hline
		B$_4$C-90                          & $14.74\pm0.09$ $\$$ & $12.28\pm0.12$ $\$$ &$1.01\pm0.09$ $\$$ \\
		Eu$_2$O$_3$                       &  $12.93\pm0.09$ $\$$    &  $10.47\pm0.12$ $\$$   &$-0.8\pm0.09$ $\$$\\
		HfB$_2$        				 &$12.62\pm0.24$ $\$$ &$10.16\pm0.26$ $\$$ &$-1.11\pm0.24$ $\$$   \\
		HfH$_{1.62}$							& $11.58\pm0.19$ $\$$ &$9.12\pm0.22$ $\$$ &$-2.15\pm0.19$ $\$$ \\
		Gd$_2$O$_3$	  		& $10.35\pm0.22$ $\$$ &$7.89\pm0.25$ $\$$& $-3.38\pm0.22$ $\$$\\
		\hline
	\end{tabularx}
	\label{tab:table2}
\end{table}

\subsubsection{Interference between CR systems}

The amplification factor (A$_{CRi}$) results show that the CSD, SSD, CSD inner 
ring, and SSD1 are slightly amplified due to the anti-shadowing effects. The 
anti-shadowing is observed when the combined rod worth is greater than the sum 
of the individual worths. The strongest anti-shadowing effect has occurred in 
SSD4 and CSD9 clusters that are located at the boundary between the core zones 
with different moderator-to-fuel ratio (see Table~\ref{tab:worth}).
The obtained results emphasize the absence of the effect of the absorption material on the interference between the control rod clusters.

Insertion of the control rod affects the neutron flux distribution, which is 
the primary reason for the amplification of CR worths indicated in 
Table~\ref{tab:worth}. Figure~\ref{fig:totalflux} illustrates the radial 
neutron flux distribution at the mid-core level with different CRs position: 
(1) all CRs withdrawn, (2) all CRs inserted, (3) all CSD inserted, (4) all SSD 
inserted. We chose the B$_4$C-90 as absorbing materials because of its high 
absorption ability. As shown in Figure~\ref{fig:totalflux}, the insertion of 
CRs deforms the radial flux shape in certain positions, i.e., around CRs 
positions. This shifts the neutron flux from the core center towards the 
periphery. The maximum neutron flux shift occurs when all CRs are inserted 
into the core.
\begin{figure}[!ht]
	\centering
	\includegraphics[width=\textwidth]{totalflux.png}
	\vspace{-0.5in}
	\caption{Radial neutron flux distribution at the mid-core for different 
	CRs positions.} 
	\label{fig:totalflux}
\end{figure}
 

\subsubsection{Integral and differential CRW}

The integral and differential control rod worth are calculated for three 
different systems: all control rods, CSD, and SSD systems. The CRs are 
inserted gradually into the core from the top to the bottom. 
Equation~\ref{Equ:4} and~\ref{Equ:5} are used to calculate the integral and 
differential CRW. Figure~\ref{fig:integ} illustrates the integral CRW for CRs 
made of B$_4$C-90. The maximum integral worth of All CRs, CSD, and SSD 
clusters are about $48.39$ , $25.3$, and $16.46$ \$, respectively. The 
integral worth of SSD clusters is sufficient to shut down the reactor from any 
state.

The differential CRWs are demonstrated in Figure~\ref{fig:diff}. Ideally, at the top of the core, the CR insertion has little effect since this region has low thermal neutron flux. Thus, the differential CRW has the lowest values in this region. The effect of CR insertion increases gradually near the center of the core. At the center of the core (region with maximum thermal neutron flux), the differential CRW is the largest and changes slowly with rod insertion. From the center of the core to the bottom, the differential CRW values decrease (region with low thermal neutron flux). Figure~\ref{fig:diff} shows that the maximum differential CRW  is shifted toward the bottom of the core. This because (Need to find a reason(s)).

Figure~\ref{fig:CSD} shows the integral CRW for only CSD clusters with five 
different absorbing materials. The results show that all absorbing materials 
have almost the same integral rod worth in the upper quarter of the core 
($<130$cm from the upper boundary of the core). Further insertion of the 
control rods shows the unique absorption characteristics of each material (see 
part~\ref{CR_worth}). B$_4$C-90 absorbs much more neutrons than the 
Gd$_2$O$_3$, which has the lowest absorption ability among the other absorbing 
materials in this study. All result are based on steady state calculations. 
Further and detailed analysis including burnup calculation will be represented 
in the near future.
\begin{figure}
	\centering
	\includegraphics[width=\textwidth]{integ.png}
	\vspace{-0.5in}
	\caption{Integral control rod worth of all CRs, CSD, and SSD clusters.} 
	\label{fig:integ}
\end{figure}
\begin{figure}
	\centering
	\includegraphics[width=\textwidth]{diff.png}
	\vspace{-0.5in}
	\caption{Differential control rod worth of all CRs, CSD, and SSD clusters.} 
	\label{fig:diff}
\end{figure}
\begin{figure}
	\centering
	\includegraphics[width=\textwidth]{CSD.png}
	\vspace{-0.5in}
	\caption{Integral control rod worth of CSD clusters for different absorbing materials.} 
	\label{fig:CSD}
\end{figure}